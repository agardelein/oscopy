\documentclass[a4paper,11pt]{article}

\usepackage[english]{babel}
\usepackage{a4wide}
\usepackage{graphicx}
\usepackage{wasysym}
\usepackage[font=small,labelfont=sf,textfont=sf]{caption}
\usepackage{hyperref}

\newcommand{\att}[1]{\texttt{#1}}
\newcommand{\meth}[1]{\texttt{#1()}}
\newcommand{\cls}[1]{\textsf{#1}}
\newcommand{\prop}[1]{\texttt{#1}}

\newcommand{\ctx}{\cls{Context}}
\newcommand{\sig}{\cls{Signal}}
\newcommand{\rd}{\cls{Reader}}
\newcommand{\rderr}{\cls{ReadError}}
\newcommand{\wrt}{\cls{Writer}}
\newcommand{\wrterr}{\cls{WriteError}}
\newcommand{\module}[1]{\textsc{#1}}
\newcommand{\graph}{\cls{Graph}}
\newcommand{\fig}{\cls{Figure}}
\newcommand{\cursor}{\cls{Cursor}}
\title{{\sc ioscopy}\\An interactive program for viewing electrical simulation results\\User Manual}
\author{Arnaud Gardelein}

\hypersetup{
colorlinks=true,
linkcolor=black,
anchorcolor=black,
citecolor=black,
filecolor=black,
menucolor=black,
pagecolor=black,
urlcolor=black,
}

\begin{document}

\maketitle
\begin{abstract}
% data plotter
% post-processing
% result viewer
% primarily targeted for electrical simulation results
% can be extended beyond that
% update and dependency tracking 
% which aims to simplify the development workflow
% interactive

Oscopy is an interactive oscilloscope written in python designed to simplify the electrical design workflow.
It allow to read, view and post-process signals with support for automatic dependency tracking.
File re-reading (updates) can be triggered by external applications like gEDA suite through D-Bus messaging system, and then Oscopy can call netlist generator and electrical simulator programs automatically.
As oscopy is built on top of IPython, post-processing include as well as simple arithmetics operation as complex functions like FFT.
Oscopy can be easily extended to a multi-purpose viewer, as adding new data file formats and new types of plots is really easy.


This document covers the user interface and command description.
% TODO: Update the abstract including short description of ioscopy and GUI
\end{abstract}

\section{Introduction}
\label{sec:intro}
% >       * create a central design document that lists all important
% >         concepts/classes (Signal, Figure, ...) and explains interactions
% >         between them; this would really help a lot
% This one is more important. When we figure out and specify how various
% object interact between themselves, we're practically done. Then all the
% code just follows naturally.
In the electrical system design workflow, viewing results from analog simulation or experi\-ment is not a trivial task: there exist numerous different program with even more different file formats, the user interface has to be friendly and functional, and the program should be memory efficient due to the number of data points per file that can quickly grow.

The gEDA suite contains mainly all tools required to design electrical boards, from scheme drawings to PCB routing.
There already exist several programs to view analog simulation results: gwave, GSpiceUI, dataplot.

Gwave is designed as a waveform viewer, an can read text file as well as binary file generated by Spice2, Spice3, ngspice, CAzM or gnucap.
The user interface present features such as drag and drop signal into the graphs, vertical bar cursors, support for multiple files and multiples panels.

GSpiceUI is more focused on the user interaction between the user and the simulation program: it import the schematic from gschem, allow the user to build the file to be used by the simulation and plot the results, eventually using GWave.

Dataplot has support for format like gnucap, ngspice, hdf5 and touchstone.
The user interface has a tabs for multiple plots, and present the data in a hierarchical manner.

Another way of viewing results is to use Octave (and generally gnuplot).
This approach permit to post-process the results with operation such as FFT, diff.
Support for multiple figures is present.
Octave support HDF5 file format and tab-separated text-based files such as gnucap output.
The user interaction is essentially based on command line interface.

The idea behind Oscopy is to combine the better of those approaches into a single program easily extendable.
In this purpose, it present features like multiple plots, multiple windows, different plot types (linear, log) and allow the user to do math with data, including basic operations, trigonometry, fft, diff.
It support the gnucap file format for input and output, and has an update mechanism to reread data from files.
New file formats and new graph types can be added by following the guidelines presented in this document.

\section{IOscopy: Oscopy on top of IPython}

\subsection{Purpose}
% TODO: Why using ipython: CLI already there, oscopy as a python module, power of ipython behind, flexibility

\subsection{Commands}
% TODO: Detailled description of ioscopy commands with an example for each one, assuming the demo/demo.oscopy script conditions

\section{Oscopy GUI}
% TODO: Description of the GUI interface, main window, menus, contextual menus, buttons




\end{document}

%%% Local Variables:
%%% Local IspellDict: british
%%% mode: latex
%%% TeX-master: t
%%% End:
