\documentclass[a4paper,11pt]{article}

\usepackage[english]{babel}
\usepackage{a4wide}
\usepackage{graphicx}
\usepackage{wasysym}
\usepackage[font=small,labelfont=sf,textfont=sf]{caption}
\usepackage{hyperref}
\usepackage[sf]{titlesec}

\newcommand{\att}[1]{\texttt{#1}}
\newcommand{\meth}[1]{\texttt{#1()}}
\newcommand{\cls}[1]{\textsf{#1}}
\newcommand{\prop}[1]{\texttt{#1}}

\newcommand{\ctx}{\cls{Context}}
\newcommand{\sig}{\cls{Signal}}
\newcommand{\rd}{\cls{Reader}}
\newcommand{\rderr}{\cls{ReadError}}
\newcommand{\wrt}{\cls{Writer}}
\newcommand{\wrterr}{\cls{WriteError}}
\newcommand{\module}[1]{\textsc{#1}}
\newcommand{\graph}{\cls{Graph}}
\newcommand{\fig}{\cls{Figure}}
\newcommand{\cursor}{\cls{Cursor}}
\title{{\sc ioscopy}\\An interactive program for viewing electrical simulation results\\User Manual}
\author{Arnaud Gardelein}

\hypersetup{
colorlinks=true,
linkcolor=black,
anchorcolor=black,
citecolor=black,
filecolor=black,
menucolor=black,
pagecolor=black,
urlcolor=black,
}

\begin{document}
\sf
\maketitle
\begin{abstract}
% data plotter
% post-processing
% result viewer
% primarily targeted for electrical simulation results
% can be extended beyond that
% update and dependency tracking 
% which aims to simplify the development workflow
% interactive

Oscopy is an interactive oscilloscope written in python designed to simplify the electrical design workflow.
It allow to read, view and post-process signals with support for automatic dependency tracking.
File re-reading (updates) can be triggered by external applications like gEDA suite through D-Bus messaging system, and then Oscopy can call netlist generator and electrical simulator programs automatically.
As oscopy is built on top of IPython, post-processing include as well as simple arithmetics operation as complex functions like FFT.
Oscopy can be easily extended to a multi-purpose viewer, as adding new data file formats and new types of plots is really easy.


This document covers the user interface and command description.
% TODO: Update the abstract including short description of ioscopy and GUI
\end{abstract}

\section{Introduction}
\label{sec:intro}
% >       * create a central design document that lists all important
% >         concepts/classes (Signal, Figure, ...) and explains interactions
% >         between them; this would really help a lot
% This one is more important. When we figure out and specify how various
% object interact between themselves, we're practically done. Then all the
% code just follows naturally.
In the electrical system design workflow, viewing results from analog simulation or experi\-ment is not a trivial task: there exist numerous different program with even more different file formats, the user interface has to be friendly and functional, and the program should be memory efficient due to the number of data points per file that can quickly grow.

The gEDA suite contains mainly all tools required to design electrical boards, from scheme drawings to PCB routing.
There already exist several programs to view analog simulation results: gwave, GSpiceUI, dataplot.

Gwave is designed as a waveform viewer, an can read text file as well as binary file generated by Spice2, Spice3, ngspice, CAzM or gnucap.
The user interface present features such as drag and drop signal into the graphs, vertical bar cursors, support for multiple files and multiples panels.

GSpiceUI is more focused on the user interaction between the user and the simulation program: it import the schematic from gschem, allow the user to build the file to be used by the simulation and plot the results, eventually using GWave.

Dataplot has support for format like gnucap, ngspice, hdf5 and touchstone.
The user interface has a tabs for multiple plots, and present the data in a hierarchical manner.

Another way of viewing results is to use Octave (and generally gnuplot).
This approach permit to post-process the results with operation such as FFT, diff.
Support for multiple figures is present.
Octave support HDF5 file format and tab-separated text-based files such as gnucap output.
The user interaction is essentially based on command line interface.

The idea behind Oscopy is to combine the better of those approaches into a single program easily extendable.
In this purpose, it present features like multiple plots, multiple windows, different plot types (linear, log) and allow the user to do math with data, including basic operations, trigonometry, fft, diff.
It support the gnucap file format for input and output, and has an update mechanism to reread data from files.
New file formats and new graph types can be added by following the guidelines presented in this document.

\section{IOscopy: Oscopy on top of IPython}

%\subsection{Purpose}
% TODO: Why using ipython: CLI already there, oscopy as a python module, power of ipython behind, flexibility

\subsection{Commands}
% TODO: Detailled description of ioscopy commands with an example for each one, assuming the demo/demo.oscopy script conditions
This section describes the ioscopy commands. Unless otherwise noticed, examples assume that demo/demo.oscopy has been run.

\newcommand{\ocmd}[2]{\vspace{5mm}\noindent\textbf{#1} #2\\}

\ocmd{oadd}{SIG [, SIG [, SIG]...]}
   Add a graph to the current figure. Figure and graph are instanciated if not present.

\begin{verbatim}
   oscopy> oadd vgs
\end{verbatim}

\ocmd{ocreate}{[SIG [, SIG [, SIG]...]]}
   Create a new figure, set it as current, add the signals in a first graph.

\begin{verbatim}
   oscopy> ocreate vgs,vds
\end{verbatim}

\ocmd{ocontext}{\ }
   Return the Context object used within ioscopy. Use it only if you want to have direct access to internal ioscopy objects.

\ocmd{odelete}{GRAPH\#}
   Delete a graph from the current figure.

\begin{verbatim}
   oscopy> odelete 1
\end{verbatim}

\ocmd{odestroy}{FIG\#}
   Destroy a figure

\begin{verbatim}
   oscopy> odestroy 3
\end{verbatim}

\ocmd{oexec}{FILENAME}
   Execute commands from file.

   This following example assume that demo/demo.oscopy has \textbf{not} been run.

\begin{verbatim}
   oscopy> oexec demo/demo.oscopy
\end{verbatim}

\ocmd{ofactors}{X, Y}
   Set the scaling factor of the graph (in power of ten). Use \texttt{auto} for automatic scaling factor.

\noindent   The following example set the scale factor at 1e-3 for X axis and 10e6 for Y axis
\begin{verbatim}
   oscopy> ofactor -3, 6
\end{verbatim}

\ocmd{ofiglist}{\ }
   Print the list of figures

\ocmd{ofreeze}{SIG [, SIG [, SIG]...]}
   Do not consider signal for subsequent updates. See also \textbf{ounfreeze}.

\ocmd{ogui}{\ }
   Show the GUI window if it was closed.

\ocmd{oimport}{SIG [, SIG [, SIG]...]}
   Import a list of signals into oscopy to handle dependencies during updates
   Example:
\begin{verbatim}
   oscopy> oread demo/trans.dat
   oscopy> pwr=iRD*vds
   oscopy> oimport pwr
   oscopy> oadd pwr
   oscopy> oupdate  #if iRD or vds changed, pwr will be automatically updated
\end{verbatim}

\ocmd{oinsert}{SIG [, SIG [, SIG]...]}
   Insert a list of signals into the current graph
\begin{verbatim}
   oscopy> oinsert vgs
\end{verbatim}

\ocmd{olayout}{horiz$|$vert$|$quad}
   Define the layout of the current figure
   \begin{description}
   \item[olayout horiz] Graphs are side by side
   \item[olayout vert] Graphs are stacked
   \item[olayout quad] One graph per figure corner
   \end{description}
\begin{verbatim}
   oscopy> olayout horiz
\end{verbatim}

\ocmd{omode}{MODE}
   Set the type of the current graph of the current figure\\
   Available modes :
   \begin{description}
   \item[omode lin]      Linear graph
   \end{description}

\ocmd{orange}{[x$|$y min max]$|$[xmin xmax ymin ymax]$|$[reset]}
   Set the axis range of the current graph of the current figure
   \begin{description}
   \item[orange x xmin xmax] set x axis range
   \item[orange y ymin ymax] set y axis range
   \item[orange xmin xmax ymin ymax] set both axis range
   \item[orange reset] set automatic range on both axis
   \end{description}
\begin{verbatim}
   oscopy> orange x -10 15
   oscopy> orange -11 20 -3.2 4.2
\end{verbatim}

\ocmd{oread}{DATAFILE}
   Read signal file
\begin{verbatim}
   oscopy> oread demo/tran.dat
\end{verbatim}

\ocmd{orefresh}{FIG\#$|$current$|$all$|$on$|$off}
   Force/toggle autorefresh of current/\#/all figures on update
   \begin{description}
   \item[orefresh FIG\#] refresh figure \#
   \item[orefresh current] refresh current figure
   \item[orefresh all]  refresh all figures
   \item[orefresh on] turn on autorefresh on Signal updates
   \item[orefresh off] turn off autorefresh on Signal updates
   \end{description}

\begin{verbatim}
   oscopy> orefresh 3
   oscopy> orefresh on
\end{verbatim}

\ocmd{oremove}{SIG [, SIG [, SIG]...]}
   Delete a list of signals into from current graph
\begin{verbatim}
   oscopy> oremove vds,vgs
\end{verbatim}

\ocmd{oscale}{[lin$|$logx$|$logy$|$loglog]}
   Set the axis scale
   \begin{description}
   \item[oscale lin] Set linear scale on both axis
   \item[oscale logx] Set log scale on x axis and linear scale on y axis
   \item[oscale logy] Set linear scale on x axis and log scale on y axis
   \item[oscale loglog] Set log scale on both axis
   \end{description}

\begin{verbatim}
   oscopy> oscale logx
\end{verbatim}

\ocmd{oselect}{FIG\#-GRAPH\#}
   Select the current figure and the current graph
\begin{verbatim}
   oscopy> oselect 2-1
\end{verbatim}

\ocmd{osiglist}{\ }
   List loaded signals
\begin{verbatim}
   oscopy> siglist
\end{verbatim}

\ocmd{ounfreeze}{SIG [, SIG [, SIG]...]}
   Consider signal for subsequent updates. See also \textbf{ofreeze}
\begin{verbatim}
   oscopy> ounfreeze vout,vds
\end{verbatim}

\ocmd{ounit}{[XUNIT,] YUNIT}
   Set the unit to be displayed on graph axis
   \begin{description}
   \item[ounit XUNIT, YUNIT] Set both axis unit
   \item[ounit YUNIT] Set Y axis unit
   \end{description}

\begin{verbatim}
   oscopy> ounit V     # Set Y axis unit
   oscopy> ounit Hz, V # Set both axis unit
\end{verbatim}

\ocmd{oupdate}{\ }
   Reread data files.

\begin{verbatim}
   oscopy> oupdate
\end{verbatim}

\ocmd{owrite}{format [(OPTIONS)] FILE SIG [, SIG [, SIG]...]}
   Write signals to file.

   This example write Signals v1 and vsqu in file \texttt{demo/res.dat} using format gnucap format, overwrite file if already existing.
\begin{verbatim}
   oscopy> owrite gnucap (ow:1) demo/res.dat v1,vsqu
\end{verbatim}

\section{Oscopy GUI}
% TODO: Description of the GUI interface, main window, menus, contextual menus, buttons




\end{document}

%%% Local Variables:
%%% Local IspellDict: british
%%% mode: latex
%%% TeX-master: t
%%% End:
